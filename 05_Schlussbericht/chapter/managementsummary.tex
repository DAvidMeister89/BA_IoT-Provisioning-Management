\chapter*{Management Summary}
\section*{Ausgangslage}
Internet of Things wird in Unternehmen immer wichtiger. Mit Hilfe von Sensoren werden bei IoT-Anwendungen reale Objekte miteinander verbunden. Bewährte Konzepte der Informatik werden angewandt, um intelligente Systeme von vernetzten Objekten zu erschaffen. In der Industrie werden bereits heute eine grosse Anzahl unterschiedlicher Sensoren eingesetzt. Vieles scheint bei IoT auf den ersten Blick bekannt, es gibt jedoch einige technische und organisatorische Unterschiede zu beachten. IoT-Devices sind verglichen mit herkömmlichen PCs und Notebooks in vielerlei Hinsicht eingeschränkt. Sie verfügen beispielsweise über weniger Ressourcen wie Rechenleistung, Speicher und Netzwerkbandbreite. Des Weiteren werden Sensoren in einer sehr viel grösseren Anzahl als herkömmliche Geräte vorhanden sein. Zukünftig werden ganze Geschäftsprozesse von einer funktionierenden IoT-Infrastruktur abhängig sein, weshalb Sicherheit immer wichtiger wird.

In einem Firmenumfeld sollten Sensoren und IoT-Devices effizient und einheitlich ausgerollt werden. Weitere Verwaltungsaufgaben wie Patching, Monitoring oder Troubleshooting müssen auch im IoT-Umfeld berücksichtigt werden. Ziel dieses Projekts ist eine umfassende Analyse über wichtige Aspekte im Bezug auf Device Management im IoT-Umfeld. Bestehende Ansätze und Lösungen sollen analysiert und erarbeitet werden, um einen geeigneten Prototypen erstellen zu können. 
\section*{Vorgehen/Technologien}
Zu Beginn hat sich das Projektteam ein Basisverständnis im IoT-Umfeld verschafft. Wichtige Aspekte des Device Managements wurden erarbeitet und aufgezeigt. Damit viele Geräte effizient verwaltet werden können, muss eine IoT-Management Applikation Funktionen für automatisches und sicheres Provisioning, Konfigurationsmanagement, Updating und Monitoring bieten. 

Kommunikation über Netzwerke ist, wie der Name verrät, bei Internet of Things ein zentrales Thema. In dieser Arbeit wird dem Leser eine Übersicht über die IoT-Architektur, sowie über Kommunikationsmodelle vermittelt. Aufgrund der geänderten Anforderungen sind neue Protokolle entstanden. Es hat sich gezeigt, dass für Geräte mit eingeschränkten Ressourcen Alternativen gewählt werden sollten. Die Open Mobile Alliance (OMA) hat mit dem ''Lightweight Machine-to-Machine"-Protokoll (LwM2M) ein massgeschneidertes Protokoll für Device Management im IoT-Umfeld entwickelt. Bereits existierende Libraries für Implementierungen des LwM2M Stacks konnten für dieses Projekt verwendet werden.

Eine breite Security-Analyse hat die Notwendigkeit von Sicherheitsüberlegungen gezeigt. Gegenüber bestehenden Infrastrukturen könnten zukünftige IoT-Systeme weitaus anfälliger werden. Ausfälle und Kompromittierungen solcher Systeme könnten drastischere Auswirkungen haben als bei heutigen Systemen. 

Aufgrund der Erkenntnisse hat das Projektteam Anforderungen an die Management Applikation definiert. Eine Architektur wurde erarbeitet und geeignete Technologien ausgewählt. Der Prototyp hat gezeigt, dass die gewählte Lösung mit einem Java-basierten LwM2M Server funktioniert. Durch das Web-Front-End können Benutzer über einen gängigen Webbrowser auf die Applikation zugreifen.
\section*{Ergebnisse}
Mit dem ''Smartmanager'' ist en funktionsfähiger Prototyp einer IoT-Management-Applikation entstanden. Durch den Einsatz des LwM2M Protokolls können theoretisch viele unterschiedliche IoT-Devices verwaltet werden. Der Benutzer hat die Möglichkeit, Konfigurationen für Devices anzulegen und diese zu verteilen. Die implementierte Gruppenverwaltung für Devices ermöglicht den Aufbau einer gewünschten Struktur. Die durch LwM2M unterstützen Operationen ''Read'', ''Write'' und ''Execute'' können sowohl einzeln auf Devices ausgeführt-, als auch auf Gruppenebene ausgeführt werden. Neue Devices können über die Discovery-Funktion dem System hinzugefügt werden. Dabei können initiale Konfigurationen und Gruppenzugehörigkeit bestimmt werden.
\section*{Ausblick}
IoT-Device-Management wird für viele Unternehmen ein Thema werden. Es ist erfreulich, dass die OMA mit LwM2M ein Protokoll entwickelt hat, um eine einheitliche Managementlösung für IoT-Devices zu ermöglichen. Es bleibt zu hoffen, dass viele Hersteller im IoT-Umfeld dieses Protokoll implementieren werden.

Wie zu erwarten war, konnten in diesem Projekt zeitlich nicht alle wichtigen Aspekte berücksichtigt werden. Für die produktive Nutzung des Tools könnten noch wichtige Sicherheitsfeatures und automatisiertes Bootstrapping implementiert werden.