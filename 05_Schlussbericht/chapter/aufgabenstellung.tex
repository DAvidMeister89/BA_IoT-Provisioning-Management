\chapter*{Aufgabenstellung}
Das ''Internet of Things'' (IoT) ist gerade am entstehen. In verschiedensten Projekten installiert die Schweizer Wirtschaft tausende von Sensoren, die alles mögliche detektieren, messen, überwachen und die Resultate in die Cloud schicken. ABER: schon nach kurzer Zeit stehen diese Firmen vor der Frage, wie eine solche Flut von Geräten provisioniert, administriert, überwacht, troubleshooted, updated und sicher ''retired'' werden kann. 

Diese Arbeit soll den Stand der Entwicklungen in diesem Umfeld aufzeigen, bestehende Lösungen evaluieren und einen eigenen Prototypen entwickeln. In einem ersten Schritt sollen bestehende Cloud Lösungen (Amazon, Azure, Google etc.) untersucht und daraus eine Zielarchitektur definiert werden. Zudem stellt sich die Frage, wie wir mit der fehlenden physical Security umgehen.

Vorgehen:
\begin{itemize}
\item Analyse von Cloud IoT Frameworks (Google, Azure, Amazon usw.)
\item Kennenlernen einer Auswahl von IoT Sensoren
\item Security-Analyse und Konzept
\item Entwicklung einer skalierbaren Management-Architektur
\item Bau eines Prototypen
\end{itemize}



\vspace{1,5 cm} 
\begin{tabular}{p{7cm}p{.5cm}l}
\dotfill \\
Prof. Beat Stettler
\end{tabular}% 
\hfill 
\begin{tabular}{p{7cm}p{.5cm}l}
\dotfill \\ 
Urs Baumann
\end{tabular}% 