\chapter{Ergebnisse}
In diesem Kapitel sind die Projektergebnisse zusammengefasst. 
\section{Analyse}
In der Analyse wurde eine breite Einarbeitung in das IoT Themengebiet erarbeitet. Es hat sich gezeigt, dass sich IoT Devices in vieler Hinsicht von herkömmlichen PC's und Notebooks unterscheiden. Es sind neue Protokolle entstanden, um den eingeschränkten Verhältnissen im IoT Umfeld gerecht zu werden.

Wichtige Aufgaben des Device Managements wurden in Kapitel 2 erarbeitet. Durch die grosse Anzahl an neuen Sensoren ist Automatisierung gefragter denn je. Wichtige Prozesse wie Bootstrapping, Monitoring, Updating und Konfiguration müssen im IoT Bereich skalierbar aufgebaut sein.

Mit LwM2M ist ein massgeschneidertes IoT Device Management Protokoll entstanden. Durch die Funktionen Bootstrapping, Registration, Resource-Access und Reporting können viele wichtige Device Management Aufgaben mit diesem Protokoll erledigt werden. Vor allem der standardisierte Ressourcenzugriff hat sich als wertvoll erwiesen. Es muss somit nicht mehr auf devicespezifische Implementierungen geachtet werden.

Die Security Analyse hat gezeigt, dass zukünftig die Bedeutung der Sicherheit noch weiter steigen wird. Die unterschiedlichen Bereich im IoT Umfeld wurden analysiert und wichtige, sicherheitsrelevante Aspekte erarbeitet.       
\section{Smartmanager}

\section{Security}