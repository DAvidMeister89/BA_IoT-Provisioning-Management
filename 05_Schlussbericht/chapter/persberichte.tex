\chapter{Persönliche Berichte}

\section{Andreas Stalder}
Das Thema Internet of Things beeindruckt mich seit langem sehr. Dies ist eine komplett neue Dimension von Netzwerken und der Anzahl Geräte, welche miteinander verbunden sind. Daher freute es mich, dass Prof. Beat Stettler eine solche Arbeit angeboten hat. Zudem konnten wir auf eine gute Zusammenarbeit bei der Studienarbeit zurückblicken, was den Entschluss diese Arbeit anzugehen befestigte.

Kurz nach dem Start der Arbeit wurde mir klar, dass ich eigentlich noch keine Ahnung im Bereich Internet of Things hatte. Das reine Interesse half nicht wirklich und so ging es an die Recherche und das Einarbeiten in die Thematik. Mir war nicht bewusst, dass dies eine so aufwendige Arbeit ist. Da das Thema Internet of Things noch sehr neu ist, gibt es keine Standards. Das heisst, es werden viele Ansätze probiert und vorgestellt. Dementsprechend gibt es auch viele verschiedene Informationen und die Recherche ist enorm aufwendig. Trotz alledem fanden wir mit LwM2M ein hilfreiches Protokoll und konnten so mit der Planung der Software starten.

Anfangs mussten wir noch viel über die eingesetzten Framworks lernen. So kannten wir Spring bis dahin nur 





\newpage
\section{David Meister}
Nachdem wir in dieser Teamkonstellation bereits die Studienarbeit absolviert haben, war für mich klar, auch die Bachelorarbeit bei Prof. Beat Stettler und Urs Baumann durchzuführen. Meine fachlichen Schwerpunkte lagen bisher eher im Bereich der Netzwerke und Security, trotzdem interessierte ich mich zu Beginn sehr für die Thematik ''Internet of Things'', welche mir bis zu dieser Arbeit fast gänzlich unbekannt war.

Der Einstieg in das Projekt gestaltete sich aus meiner Sicht schwierig. Zum einen hatte ich ein sehr grosses Interesse, war mit der Recherche jedoch häufig überfordert. Man merkt schnell, dass Internet of Things noch sehr jung ist und deshalb viele unabhängige Entwicklungen im Gange- und wenige Standards vorhanden sind. Nach einigen Wochen konnten wir jedoch spezifischer auf IoT-Protokolle eingehen und somit die zu Beginn grossen Unsicherheiten etwas beiseite legen konnten.

Beim Spezifizieren der Anforderungen wurde schnell klar, dass wir ein ''richtiges'' Software-Entwicklungsprojekt durchführen müssen. Da ich selbst ausserhalb des Studiums nie programmiert habe und deshalb nur sehr wenig praktische Erfahrung vorweisen kann, waren die Evaluation von geeigneten Technologien und Architekturen sehr schwierig für mich. Bei der Recherche und dem Erstellen des Prototyps lernte ich jedoch sehr viel. Anfangs musste ich viel Zeit investieren, um einen Einstieg in das Spring Framework zu finden. Es machte mir jedoch grossen Spass, eine Web-Applikation von Grund auf aufzubauen.

Dank der Verwendung des LwM2M-Protokolls konnten wir weitaus mehr Funktionen implementieren, als ich es je für möglich gehalten habe. Die Implementation insgesamt war aufwendig. Man konnte aber schnell gute Fortschritte erkennen, was mich sehr motiviert hat.

Verglichen mit der Studienarbeit war sowohl die Implementation, als auch die Dokumentation bei dieser Bachelorarbeit weitaus aufwendiger, was ich oft unterschätzt habe. Insgesamt kann ich mir jedoch kein schlechtes Zeitmanagement vorwerfen, den Mehraufwand hätte ich Rückblickend kaum verhindern können, ohne wichtige Dinge wegzulassen.

Mit dem Ergebnis bin ich sehr zufrieden. Die Code-Qualität müsste an vielen Stellen noch verbessert werden, was zeitlich jedoch nicht mehr möglich war. Für einen Prototyp ist die Applikation bereits erstaunlich stabil, bietet viele Funktionen und ein User Interface. Meiner Meinung nach würde es sich lohnen, die Arbeit weiter auszubauen und die Entwicklung von LwM2M weiterzuverfolgen.

Obschon es eine sehr anstrengende und schwierige Zeit war, bin ich froh, diese Arbeit gewählt zu haben. Ich habe sehr viel über Internet of Things und Web-Technologien gelernt, was mir später bestimmt zu Gute kommen wird.

Mit meinem Teamkollegen Andreas Stalder hat die Zusammenarbeit wie erwartet hervorragend geklappt. Wir haben praktisch die gesamte Zeit zusammen am gleichen Ort gearbeitet, was vieles erleichtert hat. Sobald jemand Probleme hatte oder eine Entscheidung gefällt werden musste, konnte man dies direkt anschauen und besprechen. 

Ich möchte mich in diesem Sinne besonders bei unseren Examinatoren Prof. Beat Stettler und Urs Baumann für die Betreuung dieser Arbeit bedanken. Wir wurden in jeder Phase des Projekts mit offenen Türen empfangen und ehrlich und kompetent beraten.