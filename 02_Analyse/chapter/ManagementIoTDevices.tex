\chapter{IoT Device Management}
In Zukunft ist eine stark ansteigende Anzahl an IoT Devices zu erwarten. Laut der International Data Corporation (IDC) dürften im Jahre 2020 in etwa 30 Milliarden Devices weltweit verbunden sein \cite{IDC15}. Unternehmen könnten potenziell mehrere Tausend Sensoren für ihre Zwecke einsetzen. Bereits bei herkömmlichen Computersystemen und Servern stellt das Management eine grosse Herausforderung dar. IoT Devices dürften potenziell in einer sehr viel grösseren Anzahl 
\section{Device Management Kategorien}

\subsection{Provisionierung und Authentisierung}
\subsection{Konfiguration}
\subsection{Monitoring und Diagnose}
Ein wichtiger Teil des IoT Management ist das Monitoring und die Diagnose. 


Wieso
wichtiger teil, zentral verfügbar, 





Proactive problem detection 
Helps decrease time for troubleshooting and diagnosis 
maximize system uptime and improve productivity

Logging

Überwachung -> Compute, storage, networking
Nicht 100\% benötigt

Zentral

Security



\subsection{Maintenance und Update}






\subsection{Security Management}



\section{}




















