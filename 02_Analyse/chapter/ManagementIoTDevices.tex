\chapter{IoT Device Management}
In Zukunft ist eine stark ansteigende Anzahl an IoT Devices zu erwarten. Laut der International Data Corporation (IDC) dürften im Jahre 2020 in etwa 30 Milliarden Devices weltweit verbunden sein \cite{IDC15}. Unternehmen könnten potenziell mehrere Tausend Sensoren für ihre Zwecke einsetzen. Bereits bei herkömmlichen Computersystemen und Servern stellt das Management eine grosse Herausforderung dar. IoT Devices dürften potenziell in einer sehr viel grösseren Anzahl verbreitet sein als herkömmliche Geräte. 
\section{Device Management Kategorien}

\subsection{Provisionierung und Authentisierung}
\subsection{Konfiguration}
\subsection{Monitoring und Diagnose}
\subsubsection{Was ist Monitoring?}
Bei all den Millionen Devices ist ein Monitoring unerlässlich. Ohne ein ausgeklügeltes Monitoring kann die Überwachung von so vielen Devices sehr schnell chaotisch enden. Daher muss das Monitoring gut durchdacht sein, um sein Ziel nicht zu verfehlen. Doch was ist unser Ziel mit dem Monitoring? \newline
Das Hauptziel des Monitorings ist das proaktive Überwachen der Geräte. Dadurch verringert sich nicht nur die Zeit, die benötigt wird um den Fehler zu erkennen, sondern auch die benötigte Reparaturzeit. So kann die Uptime jedes Sensors möglichst gross gehalten werden und die Produktivität steigt. Ein weiteres Ziel ist das Erkennen von Muster. So können all die gesammelten Daten zusammengefügt werden und die Muster analysiert werden. Dies führt zu einer besseren Früherkennung und auch zu einem Know-How-Gewinn. So können zukünftige Probleme besser erkannt und schneller behoben oder sogar vermieden werden.\cite{MonZiele} \newline
Nun gibt es aber neue Hindernisse bei der Überwachung von IoT-Sensoren. Nicht nur, dass es eine grössere Anzahl zu überwachende Geräte gibt, sondern auch immer neue Protokolle, Geräte die sich verschieben (SmartCars) oder auch Probleme durch den noch eher jungen Entwicklungsstand gewisser Geräte. Um ein vernünftiges Monitoring im Bereich IoT bereitzustellen, muss man daher viel bedenken, was beim normalen Server oder Netzwerkmonitoring nicht wichtig war.
\subsubsection{Monitoring im IoT-Bereich}
Im Bereich IoT gibt es spezielle Anforderungen an das Monitoring. Durch die vielen Geräte und die dynamischen Netzwerke muss das Monitoring sehr Flexibel sein. Täglich werden neue Geräte eingeführt und alte Geräte entsorgt.



Real-time monitoring
Predictive and pre-emptive maintenance
Device performance
Multi-team collaboration
Flexible



\subsection{Maintenance und Update}
\subsection{Security Management}

\section{}




















