\chapter{Einführung in Internet of Things}
\section{Übersicht}
Das Internet der Dinge unterscheidet sich in einigen Aspekten vom klassischen Internet. End-Benutzer haben über sogenannte Terminals wie Laptops oder Smartphones über die globale Internet Infrastruktur kommuniziert \cite{MiorandiSicariPellegriniChlamtac12}. Diese Terminals wurden meist von Benutzern eingeschaltet, benutzt und wieder ausgeschaltet. Damit Geräte mit dem Internet auf sinnvolle Art und Weise kommunizieren konnten, war eine manuelle Tätigkeit von Benutzern notwendig \cite{Radovici15}. Beispiele dafür sind das Abrufen von E-Mails, Surfen im Web, Streaming von Videos oder Spielen von Online Games \cite{MiorandiSicariPellegriniChlamtac12}.

Mit \glqq Internet of Things\grqq{} (IoT) wird eine andere Philosophie verfolgt. Es gibt keine einheitliche Definition und Abgrenzung von IoT. Grundsätzlich versucht man Objekte und Gegenstände, welche im klassischen Sinne des Internets nicht berücksichtigt wurden, ans Netz anzuschliessen. Mit minimalen menschlichen Eingriffen sollen diese Geräte Daten sammeln, austauschen und aufgrund von Software und Algorithmen Entscheidungen treffen \cite{RoseEldridgeChapin15}. Man spricht im Zusammenhang von \glqq Things \grqq{} auch von \glqq Smart Devices\grqq{} oder \glqq Smart Objects\grqq .
\subsection{Smart Objects}
Smart Objects oder auch \glqq Things\grqq{} ergänzen das herkömmliche Internet um eine Vielzahl neuartiger Teilnehmer. Man ist versucht, die mit dem Internet erschaffene virtuelle Welt mit Objekten der tatsächlichen \glqq echten\grqq{} Welt zu verbinden. Der Begriff \glqq Smart\grqq{} ist seit der Erscheinung des iPhones weltweit bekannt. Er beschreibt die Fähigkeit eines Objekts mit dem Internet zu kommunizieren. 


Während Smartphones oder Smart-TVs noch als herkömmliche Internet Terminals angesehen werden können, so erweitern die Smart Objects das bisherige Internet um eine neue Art von Teilnehmer. Smart Objects lassen sich wie folgt beschreiben:
\begin{itemize}
\item	haben eine physikalische Repräsentation mit Eigenschaften wie Form und Grösse
\item	haben Mindestmass an Kommunikationsfunktionalitäten wie Request/Reply
\item	besitzen eine UID (unique identifier)
\item	haben mindestens einen Namen und eine Adresse
\item	besitzen ein Mindestmass an Rechenfähigkeiten
\item	besitzen Sensoren, um physikalische Erscheinungen wie Druck, Licht, Temperatur, etc. zu messen
\end{itemize}

Der letzte Punkt in der oberen Definition beschreibt den tatsächlichen Unterschied zu herkömmlichen Devices im Internet. Konzeptionell liegt bei IoT der Fokus mehr auf Daten und Informationen von physikalischen Objekten als bei Punkt-zu-Punkt Kommunikation von Terminals \cite{MiorandiSicariPellegriniChlamtac12}.

\newpage

\section{Einsatzgebiete}
Das ''Internet of Things'' hat viele Einsatzmöglichkeiten und wir stehen erst am Anfang. Es werden immer neue Einsatzbereich entdeckt und vorhandene optimiert und erweitert. Um die Einsatzmöglichkeiten aufzuzeigen, wird hier das Beispiel ''Gesundheitsvorsorge'' genauer gezeigt.
\subsection{Allgemein}
Die Gesundheitsvorsorge ist ein riesiger Bereich, welche alle Menschen betrifft und enorme Kosten verursacht. Heutzutage wird noch sehr viel manuell erledigt. In naher Zukunft wird sich das ändern und das Internet der Dinge wird immer wichtiger, um die bestmögliche Behandlung jedes Patienten sicherzustellen. IoT bringt noch viele weitere Verbesserungen und in Zukunft werden immer neue Einsatzgebiete gefunden. Hier wird nur ein kleiner Teil beschrieben. 
\subsection{Im Alltag}
\paragraph{Selbstüberwachung}
In den letzten Jahren ist die Selbstüberwachung zu einem riesigen Thema geworden. Firmen wie Apple oder Fitbit haben diesen Bereich stark gefördert. So gibt es heute schon für mehrere Körperdatensensoren, die alles überwachen. Ein Beispiel dafür sind die Smartwatches. Diese können bereits heute den Puls rund um die Uhr überwachen oder sie zählen die Schritte mit. So kann jeder sein eigenes Fitnessprofil von sich erstellen. Ein weiteres Beispiel findet man bei Patienten, welche Medikamente verabreicht bekommen. Da Medikamente häufig vergessen werden, gibt es kluge Medikamentenspender, welche den Patienten benachrichtigen, wenn dieser die Medikamente nicht genommen hat.
\subsubsection{Fernüberwachung}
Neben der Selbstüberwachung gibt es auch noch die Fernüberwachung. Dabei wird der Patient vom Pflegepersonal überwacht und bei kritischen Situationen kann sofort gehandelt werden. Für Rollstuhlfahrer oder Senioren gibt es zum Beispiel ein Falldetektor. Falls der Patient umfällt, reagiert der Sensor und meldet diesen Unfall direkt an das Pflegepersonal. So kann schnell reagiert und dem Patienten geholfen werden. Auch das Überwachen der Patientenwerte kann so über das Internet sichergestellt werden. Durch eine kontinuierliche Abfrage der Werte, ist die Reaktionszeit enorm geringer als beim manuellen Messen. Hierfür benötigt man Pflegepersonal vor Ort.
\subsection{Im Spital}
\subsubsection{Patientenzimmer}
Durch die komplette Überwachung des Patienten in seinem Zimmer, verkürzt sich die Reaktionszeit enorm. Der Patient wird zwar schon überwacht, aber durch die Anbindung an das Internet wären weitere Verbesserungen möglich. So kann man bei gewissen Problemen nicht nur das Pflegepersonal alarmieren, sondern zum Beispiel auch direkt benötigtes Personal aus anderen Spitälern anordnen. Auch könnte man die Messdaten aus der Fernüberwachung mit der aus dem Patientenzimmer verknüpfen und so ein besseres Patientenbild erstellen. Der Patient wäre so lückenlos überwacht und bei Problemen wären die Daten sehr hilfreich.
\subsubsection{Infrastruktur}
In Spitälern werden viele verschiedenen Geräte eingesetzt, welche momentan noch nicht miteinander kommunizieren. Durch die Verknüpfung der Geräte könnte man die Daten verschiedener Geräte miteinander kombinieren und vergleichen, um ein besseres Bild erstellen zu können. Oder auch die Wartung würde viel einfacher werden. Aus der Ferne werden alle Geräte inventarisiert, gewartet und überwacht. So wird bei einem Fehler sofort der Techniker gerufen, damit ein defektes Gerät sofort wieder Einsatzbereit gemacht wird.
 
\newpage

\section{Sensortypen}
Beim Thema ''Internet of Things'' spielen die Sensoren eine zentrale Rolle. Es gibt viele verschiedene Typen von Sensoren, welche unterschiedlichste Daten liefern und man muss daher wissen, wie man mit dem jeweiligen Sensortyp umgeht. 
\begin{figure}[H]
\centering
\includegraphics[scale=0.35]{../02_Analyse/images/sensors.jpg}
\caption{Sensortypen\cite{SensorImage}}
\end{figure}

\subsection{Temperature}
Temperatursensoren werden in vielen Gebieten eingesetzt. Häufig wird dieser Sensortyp zur Überwachung von Gebäuden eingesetzt, oder hilft bei maschinell hergestellten Esswaren die richtige Temperatur zu halten. Auch kann ein Bauer diese Sensoren verwenden, um die Bodentemperatur zu überwachen. So kann man effizienter und gewinnbringender Arbeiten, ohne selber Messungen durchzuführen.
\paragraph{Bespielgeräte}
\begin{itemize}
\item	Smarte Heizungsteuerungen in einem Haushalt
\item	Wetterstationen
\end{itemize}


\subsection{Acceleration/Tilt}
Beschleunigungs- und Lagesensoren hat wohl jeder in seiner Hosentasche. Nahezu alle neuen Smartphones haben solche eingebaut. Auch in der Autoindustrie findet man solche sehr häufig. Durch solche Sensoren kann man viele verschiedene Daten erhalten. So kann man zum Beispiel Bewegungsprofile einer Person erstellen und den Fitnesslevel bestimmen.
\subsubsection{Bespielgeräte}
\begin{itemize}
\item	Schrittzähler (z.B in Smart Watches)
\end{itemize}


\subsection{Acoustic/Sound/Vibration}%DONE
Nicht nur in der Musikbranche sind Akustik- und Soundsensoren sehr wichtig. So wird auch der Lärm in einem Gebiet oder in einer Stadt gemessen, um Verbesserungen der Lebensqualität zu erreichen. Sehr wichtig sind auch die Vibrationssensoren, welche wichtige Daten zu Unterwassererdbeben senden. So können Tsunamis immer früher erkannt werden und retten Leben.
\subsubsection{Bespielgeräte}
\begin{itemize}
\item	Erdbebenwarnsysteme
\item	Sprachsteuerungen
\end{itemize}


\subsection{Chemical/Gas}%DONE
Chemikaliensensoren werden häufig in den Städten mit viel Verkehrsaufkommen eingesetzt. Dadurch wird die Luftqualität bestimmt. Auch in Laboren ist dies ein wichtiger Sensor, um die Qualität oder Reinheit von Gasen zu messen.
\subsubsection{Bespielgeräte}
\begin{itemize}
\item	Smart City Luftüberwachung
\item	Überwachung von gasbetriebenen Geräten
\end{itemize}


\subsection{Electric/Magnetic}
Magnetische Sensoren könnten in verschiedenen Geräten verbaut werden. Zum Beispiel smarte Türschlösser könnten via solchen Sensoren geschlossen oder geöffnet werden. Auch Elektrizitätswerke können diverser solche Sensoren verwenden, um die Systeme zu überwachen.
\subsubsection{Bespielgeräte}
\begin{itemize}
\item	Smarte Schlösser
\item	Stromüberwachung
\end{itemize}


\subsection{Flow}%DONE
Für die Überwachung von Flüssen oder Wasserleitungen werden Flusssensoren verwendet. So können zum Beispiel Wasserversorger den Verbrauch jedes Haushalts über das Internet messen lassen und müssen nicht vor Ort die Zähler ablesen. Oder auch für die Überwachung von Flüssen kann dieser Sensor verwendet werden. So wird man bei zu schnellen und zu vielem Wasser vor Überschwemmungen gewarnt.
\subsubsection{Bespielgeräte}
\begin{itemize}
\item	Smarte Wasserversorgungen
\item	Überschwemmungsschutz
\end{itemize}


\subsection{Force/Load/Torque/Strain/Pressure}%DONE
Im Fitnessbereich gibt es schon seit Jahren mehrere Körperwaagen, welche solche Sensoren verwenden. Man wird gewogen und gleichzeitig sendet das Gerät allerlei Daten an den Cloud-Dienst. Auch Parksysteme oder automatische Wiegesysteme in Lagern sind Beispiele, welche bereits im Einsatz sind. Diese Sensoren sind vielfältig einsetzbar.
\subsubsection{Bespielgeräte}
\begin{itemize}
\item	Wiegesysteme/Körperwaage
\item	Parksysteme
\end{itemize}


\subsection{Humidity/Moisture}%DONE
Ein wichtiger Bestandteil der Luftqualität ist auch die Luftfeuchtigkeit. Diese wird mit diesem Typ gemessen. So können Smart Buildings die Luftfeuchtigkeit laufend messen und immer wieder optimieren. Auch in der Landwirtschaft kann so eine Automation eingeführt werden, damit die Erde immer optimal bewässert ist.
\subsubsection{Bespielgeräte}
\begin{itemize}
\item	Bewässerungsanlagen
\item	Pflanzensensoren
\end{itemize}


\subsection{Leaks/Levels}%DONE
Lecks- und Levelsensoren sind zum Beispiel in der Landwirtschaft notwendig. Die Landwirtschaft benötigt viel Wasser und man möchte unnötige Lecks vermeiden. Ein weiterer wichtiger Einsatzbereich ist die Überwachung von Flüssigkeitsständen, zum Beispiel bei Staudämmen oder in Lagersystemen.
\subsubsection{Bespielgeräte}
\begin{itemize}
\item	Leitungsüberwachungen
\item	Lagerverwaltungen
\end{itemize}


\subsection{Machine Vision / Optical Ambient Light}%DONE
Machine Vision ist ein immer wichtig werdender Sensor. Durch diese kann man den Dingen das Sehen ''lehren''. So sind automatische Eintrittkontrollen möglich. Auch in den SmartCars kommen diese Sensoren zum Einsatz. Da werden durch die Sensoren die Fussgänger, die Fahrbahn oder auch andere Autos erkannt. Bei Optical Ambient Light geht es um Sensoren, welche die Umgebungsbeleuchtung messen und diese Optimal anpassen. So gibt es schon mehrere smarte Leuchtmittel, welche so über das Internet eingestellt werden können.
\subsubsection{Bespielgeräte}
\begin{itemize}
\item	SmartCars
\item	Eintrittskontrollen
\end{itemize}


\subsection{Motion/Velocity/Displacement}%DONE
Bewegungsensoren sind praktische Helfer bei Sicherheitssystemen. Eine zentrale Überwachung von mehreren Gebäuden ist so problemlos möglich. Bei der Pflege von Rollstuhlfahrer wäre zum Beispiel auch eine Falldetektion möglich. Das zentrale Pflegezentrum hätte so eine schnelle Meldung über Unfälle und könnte mehrere Personen überwachen und betreuen.
\subsubsection{Bespielgeräte}
\begin{itemize}
\item	Sicherheitsanlagen
\item	Verkehrsüberwachung
\end{itemize}


\subsection{Position/Presence/Proximity}%DONE
Immer wichtiger wird auch dieser Typ von Sensoren. Durch die Bestimmung von Distanzen oder der Position in einem Raum, ergeben sich viele Einsatzmöglichkeiten. Ein bekanntes Beispiel ist der Abstandssensor bei Autos, um Parkschäden oder Auffahrunfälle zu vermeiden. Oder auch die Gebäudeüberwachung profitiert durch solche Sensoren, da man mehrere Gebäude zentral Überwachen kann.
\subsubsection{Bespielgeräte}
\begin{itemize}
\item	Smarte Parkhäuse
\item	SmartCars
\end{itemize}