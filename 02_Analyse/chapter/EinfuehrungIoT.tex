\chapter{Einführung in Internet of Things}
\section{Übersicht}
Das Internet der Dinge unterscheidet sich in einigen Aspekten vom klassischen Internet. End-Benutzer haben über sogenannte Terminals wie Laptops oder Smartphones über die globale Internet Infrastruktur kommuniziert \cite{MiorandiSicariPellegriniChlamtac12}. Diese Terminals wurden meist von Benutzern eingeschaltet, benutzt und wieder ausgeschaltet. Damit Geräte mit dem Internet auf sinnvolle Art und Weise kommunizieren konnten, war eine manuelle Tätigkeit von Benutzern notwendig \cite{Radovici15}. Beispiele dafür sind das Abrufen von E-Mails, Surfen im Web, Streaming von Videos oder Spielen von Online Games \cite{MiorandiSicariPellegriniChlamtac12}.

Mit \glqq Internet of Things\grqq{} (IoT) wird eine andere Philosophie verfolgt. Es gibt keine einheitliche Definition und Abgrenzung von IoT. Grundsätzlich versucht man Objekte und Gegenstände, welche im klassischen Sinne des Internets nicht berücksichtigt wurden, ans Netz anzuschliessen. Mit minimalen menschlichen Eingriffen sollen diese Geräte Daten sammeln, austauschen und aufgrund von Software und Algorithmen Entscheidungen treffen \cite{RoseEldridgeChapin15}. Man spricht im Zusammenhang von \glqq Things \grqq{} auch von \glqq Smart Devices\grqq{} oder \glqq Smart Objects\grqq.
\subsection{Smart Objects}
Smart Objects oder auch \glqq Things\grqq{} ergänzen das herkömmliche Internet um eine Vielzahl neuartiger Teilnehmer. Man ist versucht, die mit dem Internet erschaffene virtuelle Welt mit Objekten der tatsächlichen \glqq echten\grqq{} Welt zu verbinden. Der Begriff \glqq Smart\grqq{} ist seit der Erscheinung des iPhones weltweit bekannt. Er beschreibt die Fähigkeit eines Objekts mit dem Internet zu kommunizieren. 


Während Smartphones oder Smart-TVs noch als herkömmliche Internet Terminals angesehen werden können, so erweitern die Smart Objects das bisherige Internet um eine neue Art von Teilnehmer. Smart Objects lassen sich wie folgt beschreiben:
\begin{itemize}
\item	haben eine physikalische Repräsentation mit Eigenschaften wie Form und Grösse
\item	haben Mindestmass an Kommunikationsfunktionalitäten wie Request/Reply
\item	besitzen eine UID (unique identifier)
\item	haben mindestens einen Namen und eine Adresse
\item	besitzen ein Mindestmass an Rechenfähigkeiten
\item	besitzen Sensoren, um physikalische Erscheinungen wie Druck, Licht, Temperatur, etc. zu messen
\end{itemize}

Der letzte Punkt in der oberen Definition beschreibt den tatsächlichen Unterschied zu herkömmlichen Devices im Internet. Konzeptionell liegt bei IoT der Fokus mehr auf Daten und Informationen von physikalischen Objekten als bei Punkt-zu-Punkt Kommunikation von Terminals \cite{MiorandiSicariPellegriniChlamtac12}.

\newpage

\section{Einsatzgebiete}
Das ''Internet of Things'' hat viele Einsatzmöglichkeiten und wir stehen erst am Anfang. Es werden immer neue Einsatzbereich entdeckt und vorhandene optimiert und erweitert. Um die Einsatzmöglichkeiten aufzuzeigen, wird hier das Beispiel ''Gesundheitsvorsorge'' genauer gezeigt.
\subsection{Allgemein}
Die Gesundheitsvorsorge ist ein riesiger Bereich, welche alle Menschen betrifft und enorme Kosten verursacht. Heutzutage wird noch sehr viel manuell erledigt. In naher Zukunft wird sich das ändern und das Internet der Dinge wird immer wichtiger, um die bestmögliche Behandlung jedes Patienten sicherzustellen. 
\subsection{Im Alltag}
\paragraph{Selbstüberwachung}
In den letzten Jahren ist die Selbstüberwachung zu einem riesigen Thema geworden. Firmen wie Apple oder Fitbit haben diesen Bereich stark gefördert. So gibt es heute schon für mehrere Körperdatensensoren, die alles überwachen. Ein Beispiel dafür sind die Smartwatches. Diese können bereits heute den Puls rund um die Uhr überwachen oder sie zählen die Schritte mit. So kann jeder sein eigenes Fitnessprofil von sich erstellen. Ein weiteres Beispiel findet man bei Patienten, welche Medikamente verabreicht bekommen. Da Medikamente häufig vergessen werden, gibt es kluge Medikamentenspender, welche den Patienten benachrichtigen, wenn dieser die Medikamente nicht genommen hat.
\subsubsection{Fernüberwachung}
Neben der Selbstüberwachung gibt es auch noch die Fernüberwachung. Dabei wird der Patient vom Pflegepersonal überwacht und bei kritischen Situationen kann sofort gehandelt werden. Für Rollstuhlfahrer oder Senioren gibt es zum Beispiel ein Falldetektor. Falls der Patient umfällt, reagiert der Sensor und meldet diesen Unfall direkt an das Pflegepersonal. So kann schnell reagiert und dem Patienten geholfen werden. Auch das Überwachen der Patientenwerte kann so über das Internet sichergestellt werden. Durch eine kontinuierliche Abfrage der Werte, ist die Reaktionszeit enorm geringer als beim manuellen Messen. Hierfür benötigt man Pflegepersonal vor Ort.
\subsection{Im Spital}
\subsubsection{Patientenzimmer}
Durch die komplette Überwachung des Patienten in seinem Zimmer, verkürzt sich die Reaktionszeit enorm. Der Patient wird zwar schon überwacht, aber durch die Anbindung an das Internet wären weitere Verbesserungen möglich. So kann man bei gewissen Problemen nicht nur das Pflegepersonal alarmieren, sondern zum Beispiel auch direkt benötigtes Personal aus anderen Spitälern anordnen. Auch könnte man die Messdaten aus der Fernüberwachung mit der aus dem Patientenzimmer verknüpfen und so ein besseres Patientenbild erstellen.
\subsubsection{Infrastruktur}
In Spitälern werden viele verschiedenen Geräte eingesetzt, welche momentan noch nicht miteinander kommunizieren. Durch die Verknüpfung der Geräte könnte man die Daten verschiedener Geräte miteinander kombinieren und vergleichen, um ein besseres Bild erstellen zu können. Oder auch die Wartung würde viel einfacher werden. Aus der Ferne werden alle Geräte inventarisiert, gewartet und überwacht. So wird bei einem Fehler sofort der Techniker gerufen, damit ein defektes Gerät sofort wieder Einsatzbereit gemacht wird.