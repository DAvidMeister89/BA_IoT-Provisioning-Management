\chapter{Einführung in ''Internet of Things''}
\section{Übersicht}
\section{Einsatzgebiete}
''Internet of Things'' hat extrem viele Einsatzmöglichkeiten und wir stehen erst am Anfang. Es werden immer neue Einsatzbereich entdeckt und vorhandene erweitert und optimiert. Um die Einsatzmöglichkeiten aufzuzeigen, wird hier das Beispiel ''Gesundheitsvorsorge'' genauer gezeigt.
\subsection{Zuhause}
\subsection{Beim Doktor}
\subsection{Im Spital}
\subsection{•}
\section{Sensortypen}
Beim Thema ''Internet of Things'' spielen die Sensoren eine zentrale Rolle. Es gibt viele verschiedene Typen von Sensoren, welche unterschiedlichste Daten liefern und man muss daher wissen, wie man mit dem jeweiligen Sensortyp umgeht. 
\subsection{Temperature}
Temperatursensoren werden in vielen Gebieten eingesetzt. Häufig wird dieser Sensortyp zur Überwachung von Gebäuden eingesetzt, oder hilft bei maschinell hergestellten Esswaren, die richtige Temperatur zu halten. Auch kann ein Bauer diese Sensoren verwenden, um die Bodentemperatur zu Überwachen. So kann man effizienter und gewinnbringender Arbeiten, ohne selber Messungen durchzuführen. 
\subsection{Acceleration/Tilt}
Beschleunigungs- und Lagesensoren hat wohl jeder in seiner Hosentasche. Nahezu alle neuen Smartphones haben solche eingebaut. Auch in der Autoindustrie findet man solche sehr häufig. Durch solche Sensoren kann man viele Verschiedene Daten erhalten. So kann man zum Beispiel Bewegungsprofile einer Person erstellen und den Fitnesslevel bestimmen.
\subsection{Acoustic/Sound/Vibration}
Nicht nur in der Musikbranche sind Akustik- und Soundsensoren sehr wichtig. So wird auch der Lärm in einem Gebiet oder in einer Stadt gemessen, um Verbesserungen der Lebensqualität zu erreichen. Sehr wichtig sind auch die Vibrationssensoren, welche wichtige Daten zu Unterwassererdbeben senden. So können Tsunamis immer früher erkannt werden und retten Leben. 
\subsection{Chemical/Gas}
Chemikaliensensoren werden häufig in den Städten mit viel Verkehrsaufkommen eingesetzt. Dadurch wird die Luftqualität bestimmt. Auch in Laboren ist dies ein wichtiger Sensor, um die Qualität oder Reinheit von Gasen zu messen.
\subsection{Electric/Magnetic}

\subsection{Flow}
Für die Überwachung von Flüssen oder Wasserleitungen werden Flusssensoren verwendet. So können zum Beispiel Wasserversorger den Verbrauch jedes Haushalts über das Internet messen lassen und müssen nicht vor Ort die Zähler ablesen. Oder auch für die Überwachung von Flüssen kann dieser Sensor verwendet werden. So wird man bei zu schnellen und zu vielem Wasser vor Überschwemmungen gewarnt.
\subsection{Force/Load/Torque/Strain/Pressure}

\subsection{Humidity/Moisture}
Ein wichtiger Bestandteil der Luftqualität ist auch die Luftfeuchtigkeit. Diese wird mit diesem Typ gemessen. So können Smart Buildings die Luftfeuchtigkeit laufend messen und immer wieder optimieren. Auch in der Landwirtschaft kann so eine Automation eingeführt werden, damit die Erde immer optimal bewässert ist.
\subsection{Leaks/Levels}
Lecks- und Levelsensoren sind zum Beispiel in der Landwirtschaft notwendig. Die Landwirtschaft benötigt  viel Wasser und man möchte unnötige Lecks vermeiden. Ein weiterer wichtiger Einsatzbereich ist die Überwachung von Flüssigkeitsständen, zum Beispiel bei Staudämmen oder in Lagersystemen.
\subsection{Machine Vision / Optical Ambient Light}

\subsection{Motion/Velocity/Displacement}

\subsection{Position/Presence/Proximity}
Immer wichtiger wird auch dieser Typ von Sensoren. Durch die Bestimmung von Distanzen oder der Position in einem Raum, ergeben sich viele Einsatzmöglichkeiten. Ein bekanntes Beispiel ist der Abstandssensor bei Autos, um Parkschäden oder Auffahrunfälle zu vermeiden. Oder auch die Gebäudeüberwachung profitiert durch solche Sensoren, da man mehrere Gebäude zentral Überwachen kann.
\section{Kommunikation}
\subsection{HTTP}
\subsection{Weitere Protokolle}
\section{Security Aspekte}