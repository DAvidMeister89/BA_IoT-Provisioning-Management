\chapter{Requirements}
\section{Allgemeine Beschreibung}
\subsection{Produktperspektive}
Mit Internet of Things sind eine Vielzahl neuartige Devices entstanden. Während in herkömmlichen Netzwerken hauptsächlich Personal Computer, Notebooks, Server usw. verwaltet werden mussten, so bringen IoT Devices den IT-Abteilungen neue Herausforderungen. Zum einen dürfte die Anzahl Geräte gegenüber herkömmlichen Computer deutlich ansteigen, zum anderen sind IoT Devices in Sachen Funktionalität und Rechenleistung, sowie auch der Netzwerkbandbreite deutlich beschränkt. 

Mit <insert Application Name here> soll eine Management Applikation bereitgestellt werden, um eine Vielzahl unterschiedlicher IoT Devices administrieren zu können. 
\subsection{Produkfunktionen}
<insert Application Name here> soll den Benutzern erlauben, IoT Geräte zu verwalten. Die Aufgaben reichen vom Erfassen und Discovery von Devices über die Konfigurationsverwaltung und Softwareverteilung bis zu Backup und Restore. Ausserdem sollen Management-relevante Kommandos auf Devices ausgeführt- und Security Aspekte beachtet werden. Die Details zu den Produktfunktionen sind den Use Cases zu entnehmen.

\subsection{Benutzer Charakteristik}
Zielpersonen der Applikation sind Betreiber von IoT Devices. Dies können im Enterprise Umfeld IT-Mitarbeiter in operationeller Funktion-, oder auch Softwareentwickler für IoT Applikationen sein. Heimanwender können bei entsprechenden Kenntnissen ebenfalls zur Zielgruppe gehören. Es werden solide Grundkenntnisse in TCP/IP Netzwerken sowie Verständnis der verwendeten IoT Architekturen und Devices vorausgesetzt. 
\subsection{Einschränkungen}
Evtl. Browser Einschränkungen und bestimmte Use Cases, müsste nach Prototyp nochmals definiert werden
\section{Use Cases}
\subsection{Use Cases Diagramm}
%\includegraphics[scale=0.52]{figures/UseCase_Diagram}
\subsection{Aktoren}
Der Benutzer der Applikation ist in diesem System der einzige primäre Aktor.
\subsection{Beschreibungen (Casual)}
\subsubsection{Benutzer CRUD}
\mbox{}
\begin{longtable}{| p{4cm} | p{11.7cm} |}
 \hline
 \textbf{ID} & 01 \\ \hline 
 \textbf{Name} & • \\ \hline 
 \textbf{Beschreibung} & • \\ \hline 
 \textbf{Preconditions} & • \\ \hline 
 \textbf{Postconditions} & • \\ \hline 
 \textbf{Main Success Scenario} & • \\ \hline 
 \textbf{Extensions} & • \\ \hline 
 \end{longtable}

\subsubsection{Device erfassen}
\mbox{}
\begin{longtable}{| p{4cm} | p{11.7cm} |}
 \hline
 \textbf{ID} & 02\\ \hline 
 \textbf{Name} & Device erfassen \\ \hline 
 \textbf{Beschreibung} & Der Benutzer möchte ein IoT Device finden. Entsprechende IoT Devices sollen dem Benutzer zur Adoption aufgelistet werden. \\ \hline 
 \textbf{Preconditions} & • \\ \hline 
 \textbf{Postconditions} & • \\ \hline 
 \textbf{Main Success Scenario} & • \\ \hline 
 \textbf{Extensions} & • \\ \hline 
\end{longtable}

\subsubsection{Device finden}
\mbox{}
\begin{longtable}{| p{4cm} | p{11.7cm} |}
 \hline
 \textbf{ID} & 03\\ \hline 
 \textbf{Name} & • \\ \hline 
 \textbf{Beschreibung} & Der Benutzer möchte ein IoT Device manuell hinzufügen. Der Endpunkt ist dem Benutzer bekannt. \\ \hline 
 \textbf{Preconditions} & • \\ \hline 
 \textbf{Postconditions} & • \\ \hline 
 \textbf{Main Success Scenario} & • \\ \hline 
 \textbf{Extensions} & • \\ \hline 
 \end{longtable}
 
\subsubsection{Device assoziieren}
\mbox{}
\begin{longtable}{| p{4cm} | p{11.7cm} |}
 \hline
 \textbf{ID} & 04\\ \hline 
 \textbf{Name} & Device assoziieren \\ \hline 
 \textbf{Beschreibung} & Der Benutzer möchte ein- oder mehrere IoT Device(s) verwalten. Dazu muss er das gefundene Device in das System adoptieren. \\ \hline 
 \textbf{Preconditions} & • \\ \hline 
 \textbf{Postconditions} & • \\ \hline 
 \textbf{Main Success Scenario} & • \\ \hline 
 \textbf{Extensions} & • \\ \hline 
 \end{longtable}
 
\subsubsection{Device CRUD}
\mbox{}
\begin{longtable}{| p{4cm} | p{11.7cm} |}
 \hline
 \textbf{ID} & 05\\ \hline 
 \textbf{Name} & • \\ \hline 
 \textbf{Beschreibung} & • \\ \hline 
 \textbf{Preconditions} & • \\ \hline 
 \textbf{Postconditions} & • \\ \hline 
 \textbf{Main Success Scenario} & • \\ \hline 
 \textbf{Extensions} & • \\ \hline 
 \end{longtable}
 
\subsubsection{Device abfragen}
\mbox{}
\begin{longtable}{| p{4cm} | p{11.7cm} |}
 \hline
 \textbf{ID} & 06\\ \hline 
 \textbf{Name} & • \\ \hline 
 \textbf{Beschreibung} & • \\ \hline 
 \textbf{Preconditions} & • \\ \hline 
 \textbf{Postconditions} & • \\ \hline 
 \textbf{Main Success Scenario} & • \\ \hline 
 \textbf{Extensions} & • \\ \hline 
 \end{longtable}
 
\subsubsection{Konfiguration speichern}
\mbox{}
\begin{longtable}{| p{4cm} | p{11.7cm} |}
 \hline
 \textbf{ID} & 07\\ \hline 
 \textbf{Name} & • \\ \hline 
 \textbf{Beschreibung} & • \\ \hline 
 \textbf{Preconditions} & • \\ \hline 
 \textbf{Postconditions} & • \\ \hline 
 \textbf{Main Success Scenario} & • \\ \hline 
 \textbf{Extensions} & • \\ \hline 
 \end{longtable}
 
\subsubsection{Konfiguration verteilen}
\mbox{}
\begin{longtable}{| p{4cm} | p{11.7cm} |}
 \hline
 \textbf{ID} & 08\\ \hline 
 \textbf{Name} & • \\ \hline 
 \textbf{Beschreibung} & • \\ \hline 
 \textbf{Preconditions} & • \\ \hline 
 \textbf{Postconditions} & • \\ \hline 
 \textbf{Main Success Scenario} & • \\ \hline 
 \textbf{Extensions} & • \\ \hline 
 \end{longtable}
 
\subsubsection{Konfiguration anzeigen}
\mbox{}
\begin{longtable}{| p{4cm} | p{11.7cm} |}
 \hline
 \textbf{ID} & 09\\ \hline 
 \textbf{Name} & • \\ \hline 
 \textbf{Beschreibung} & • \\ \hline 
 \textbf{Preconditions} & • \\ \hline 
 \textbf{Postconditions} & • \\ \hline 
 \textbf{Main Success Scenario} & • \\ \hline 
 \textbf{Extensions} & • \\ \hline 
 \end{longtable}
 
\subsubsection{Software speichern}
\mbox{}
\begin{longtable}{| p{4cm} | p{11.7cm} |}
 \hline
 \textbf{ID} & 10\\ \hline 
 \textbf{Name} & • \\ \hline 
 \textbf{Beschreibung} & • \\ \hline 
 \textbf{Preconditions} & • \\ \hline 
 \textbf{Postconditions} & • \\ \hline 
 \textbf{Main Success Scenario} & • \\ \hline 
 \textbf{Extensions} & • \\ \hline 
 \end{longtable}
 
\subsubsection{Software verteilen}
\mbox{}
\begin{longtable}{| p{4cm} | p{11.7cm} |}
 \hline
 \textbf{ID} & 11\\ \hline 
 \textbf{Name} & • \\ \hline 
 \textbf{Beschreibung} & • \\ \hline 
 \textbf{Preconditions} & • \\ \hline 
 \textbf{Postconditions} & • \\ \hline 
 \textbf{Main Success Scenario} & • \\ \hline 
 \textbf{Extensions} & • \\ \hline 
 \end{longtable}
 
\subsubsection{Device sichern}
\mbox{}
\begin{longtable}{| p{4cm} | p{11.7cm} |}
 \hline
 \textbf{ID} & 12\\ \hline 
 \textbf{Name} & • \\ \hline 
 \textbf{Beschreibung} & • \\ \hline 
 \textbf{Preconditions} & • \\ \hline 
 \textbf{Postconditions} & • \\ \hline 
 \textbf{Main Success Scenario} & • \\ \hline 
 \textbf{Extensions} & • \\ \hline 
 \end{longtable}
 
\subsubsection{Device wiederherstellen}
\mbox{}
\begin{longtable}{| p{4cm} | p{11.7cm} |}
 \hline
 \textbf{ID} & 13\\ \hline 
 \textbf{Name} & • \\ \hline 
 \textbf{Beschreibung} & • \\ \hline 
 \textbf{Preconditions} & • \\ \hline 
 \textbf{Postconditions} & • \\ \hline 
 \textbf{Main Success Scenario} & • \\ \hline 
 \textbf{Extensions} & • \\ \hline 
 \end{longtable}
 
\subsubsection{Device authentisieren}
\mbox{}
\begin{longtable}{| p{4cm} | p{11.7cm} |}
 \hline
 \textbf{ID} & 14\\ \hline 
 \textbf{Name} & • \\ \hline 
 \textbf{Beschreibung} & • \\ \hline 
 \textbf{Preconditions} & • \\ \hline 
 \textbf{Postconditions} & • \\ \hline 
 \textbf{Main Success Scenario} & • \\ \hline 
 \textbf{Extensions} & • \\ \hline 
 \end{longtable}
 
\subsubsection{Schlüssel verwalten}
\mbox{}
\begin{longtable}{| p{4cm} | p{11.7cm} |}
 \hline
 \textbf{ID} & 14.1\\ \hline 
 \textbf{Name} & • \\ \hline 
 \textbf{Beschreibung} & • \\ \hline 
 \textbf{Preconditions} & • \\ \hline 
 \textbf{Postconditions} & • \\ \hline 
 \textbf{Main Success Scenario} & • \\ \hline 
 \textbf{Extensions} & • \\ \hline 
 \end{longtable}
 
\subsubsection{Zertifikate verteilen}
\mbox{}
\begin{longtable}{| p{4cm} | p{11.7cm} |}
 \hline
 \textbf{ID} & 14.2\\ \hline 
 \textbf{Name} & • \\ \hline 
 \textbf{Beschreibung} & • \\ \hline 
 \textbf{Preconditions} & • \\ \hline 
 \textbf{Postconditions} & • \\ \hline 
 \textbf{Main Success Scenario} & • \\ \hline 
 \textbf{Extensions} & • \\ \hline 
 \end{longtable}
 
\subsubsection{Zertifikate CRUD}
\mbox{}
\begin{longtable}{| p{4cm} | p{11.7cm} |}
 \hline
 \textbf{ID} & 14.3\\ \hline 
 \textbf{Name} & • \\ \hline 
 \textbf{Beschreibung} & • \\ \hline 
 \textbf{Preconditions} & • \\ \hline 
 \textbf{Postconditions} & • \\ \hline 
 \textbf{Main Success Scenario} & • \\ \hline 
 \textbf{Extensions} & • \\ \hline 
 \end{longtable} 
 
\subsubsection{Device Kommandos ausführen}
\mbox{}
\begin{longtable}{| p{4cm} | p{11.7cm} |}
 \hline
 \textbf{ID} & 15\\ \hline 
 \textbf{Name} & • \\ \hline 
 \textbf{Beschreibung} & • \\ \hline 
 \textbf{Preconditions} & • \\ \hline 
 \textbf{Postconditions} & • \\ \hline 
 \textbf{Main Success Scenario} & • \\ \hline 
 \textbf{Extensions} & • \\ \hline 
\end{longtable}


 
 

\section{Nichtfunktionale Anforderungen}
In diesem Kapitel behandeln wir die nichtfunktionalen Anforderungen an das Projekt. Wir behandeln Aspekte und Anforderungen aus den Bereichen Qualität, Schnittstellen und Randbedingungen.
\subsection{Qualität}
Bei der Softwarequalität stützen wir uns auf die ISO/IEC 9126 Norm. Es werden die Merkmale Funktionalität, Zuverlässigkeit, Benutzbarkeit, Effizienz, Wartbarkeit und Übertragbarkeit aufgeführt.
\subsubsection{Funktionalität}
Netzwerkdevices können von vielen unterschiedlichen Herstellern kommen. Diese Hersteller verwenden unterschiedliche Syntax und Ausgabeformate. Um die Funktionalität best möglich sicherzustellen, wird die Herstellerunterstützung vorerst stark eingeschränkt. Vorgesehen sind vorerst Cisco Netzwerkdevices und Linux Hosts.
\subsubsection{Zuverlässigkeit}
Tests sind wichtig und nützlich, jedoch nicht Business kritisch bei einem möglichen Ausfall. Es muss vor allem darauf geachtet werden, dass bei ssh Verbindungen ein sauberes Exception Handling implementiert wird, falls beim Verbindungsaufbau oder bei abgesetzten Kommandos etwas schief geht. Es müssen für gewisse Tests auch Timeouts eingeplant werden, damit das Programm nicht unendlich lange blockieren kann.
\subsubsection{Benutzbarkeit}
Wir möchten ein schmales User Interface auf Konsolen Ebene bieten. Der Anwender muss über Kenntnisse auf der Linux bash verfügen. Mittels eingebauter Hilfe soll es versierten Benutzern möglich sein, die Software zu verwenden.
\subsubsection{Effizienz}
Die Effizienz ist sehr stark von den getesteten Devices und den abgesetzten Kommandos abhängig. 
\subsubsection{Wartbarkeit}
Eigene Test Cases sollen mit den notwendigen Kenntnissen selbst ergänzt werden können. Command-Mapping und Outputs müssen bekannt sein, dann ist eine Erweiterung des Funktionsumfangs der Test Cases denkbar.
\subsubsection{Übertragbarkeit}
Die Übertragbarkeit auf andere Plattformen oder Hersteller ist schwierig und vorerst nicht vorgesehen.

\subsection{Schnittstellen}
\subsubsection{Benutzerschnittstellen}
Die Steuerung des Programms ist mittels Tastatur über die Linux bash vorgesehen.

\subsubsection{Netzwerkschnittstellen}
Im Programm können alle Devices verwendet werden, welche über ein lokales Netzwerkinterface erreichbar sind.

\subsection{Sicherheit}
Um auf Devices verbinden zu können, muss man sich auf diesen authentifizieren. Administrator Zugänge müssen deshalb best möglichst geschützt sein. Die Übertragung muss verschlüsselt sein (ssh) und die Passwörter dürfen, wenn überhaupt, nur mittels sicherem Hashverfahren abgelegt werden.